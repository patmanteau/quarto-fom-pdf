\SetTracking[
    spacing = {45*,,},
    outer spacing = {30*,,}
]{
    encoding= *,
    shape = sc
}{5}

\SetTracking[
    context = allcaps,
    spacing = {400*,,},
    outer spacing = {300*,,}
]{
    encoding = *
}{50}

%%%%%%%%%%%%%%%%%%%%%%%%%%%%%%%%%%%%%%%%%%%%%%
%%
%% Kerninganpassungen
%% s. https://tex.stackexchange.com/a/370469
%%
%%%%%%%%%%%%%%%%%%%%%%%%%%%%%%%%%%%%%%%%%%%%%%

% \directlua{
% fonts.handlers.otf.addfeature {
%     name = "cuskern",
%     type = "kern",
%     data = {
%         ["f"] = { 
%             ["1"] =  2000,
%             ["2"] =  200 
%         },
%     },
% }}

% \setmainfont{EB Garamond}[
%     Ligatures       = {Common,TeX},
%     % Scale           = 0.94, % 0.94 makes it about Times New Roman size
%     Numbers         = {OldStyle},
%      Contextuals     = {Alternate}
% ]
\setmainfont{Vollkorn}[
    Ligatures       = {Common,TeX},
    Scale           = 0.98, % 0.98 makes it about Times New Roman size
    Numbers         = {OldStyle},
     Contextuals     = {Alternate}
]
% \setmainfont{STIX}[
%     Ligatures       = {Common,TeX},
%     % Scale           = 0.94, % 0.94 makes it about Times New Roman size
%     Numbers         = {OldStyle},
%      Contextuals     = {Alternate}
% ]
% \setsansfont{Libertinus Sans}[
%     Ligatures       = {Common,TeX},
%     % Scale           = 0.94, % 0.94 makes it about Times New Roman size
%     Numbers         = {OldStyle},
%      Contextuals     = {Alternate}
% ]
\setmathfont{Garamond-Math.otf}[StylisticSet={8,9}] 


% \setmainfont{STIX Two Text}[
%     Ligatures       = {Common,TeX},
%     % Scale           = 0.94, % 0.94 makes it about Times New Roman size
%     Numbers         = {OldStyle},
%      Contextuals     = {Alternate}
% ]
% \setsansfont{Libertinus Sans}[
%     Ligatures       = {Common,TeX},
%     % Scale           = 0.94, % 0.94 makes it about Times New Roman size
%     Numbers         = {OldStyle},
%      Contextuals     = {Alternate}
% ]
% \setmathfont{STIX Two Math} 


% % Serifenschrift...
% \setmainfont{Libertinus Serif}[
%     Path            = fonts/,
%     Extension       = .otf,
%     UprightFont     = *-Regular,
%     BoldFont        = *-Semibold,
%     ItalicFont      = *-Italic,
%     BoldItalicFont  = *-SemiboldItalic,
%     RawFeature      = +cuskern,
%     Ligatures       = {Common,TeX},
%     Scale           = 0.94, % 0.94 makes it about Times New Roman size
%     Numbers         = {OldStyle},
%     % Renderer        = Harfbuzz,
%     Contextuals     = {Alternate}
% ]

% % Serifenlose...
% \setsansfont{SkolarSansPE}[
%     Path            = fonts/,
%     Extension       = .otf,
%     UprightFont     = *-Rg,
%     BoldFont        = *-Bd,
%     ItalicFont      = *-It,
%     BoldItalicFont  = *-BdIt,
%     Ligatures       = {Common, TeX},
%     Scale           = 0.94,
%     Numbers         = {OldStyle},
%     Contextuals     = {Alternate}
% ]

% % Nichtproportionale Schrift...
% \setmonofont{Iosevka}[Scale=MatchLowercase]

% % Matheschrift
% \setmathfont{Cambria Math}[
%     Ligatures=TeX,
%     Scale=MatchLowercase
% ]

% \setmathfontface\mathoper{Cambria Math}[
%     Ligatures=TeX,
%     Scale=MatchLowercase
% ]
% \setoperatorfont\mathoper

% %% Fancy Schauschrift
% \newfontface\fancyheadline{SkolarSansPE-Eb}[
%     Path            = fonts/,
%     Extension       = .otf,
%     UprightFont     = SkolarSansPE-Eb,
%     Ligatures=TeX
% ]

% %% Fancy Schauschrift
% \newfontface\fancysemibold{SkolarPE-Semibold}[
%     Path            = fonts/,
%     Extension       = .otf,
%     ItalicFont=SkolarPE-SemiboldItalic,
%     Ligatures={Common, TeX},
%     Numbers={OldStyle},
%     Contextuals={Alternate}
% ]



% Microtype konfigurieren
\iftoggle{final_version}{%
\usepackage[tracking=true,final]{microtype}
}{%
\usepackage[tracking=true]{microtype}
}


%% ----------------------------------------------------------------------------
%%
%% Zeilenabstand
%%
%% ----------------------------------------------------------------------------

%% Default ist 1,5facher Zeilenabstand
\usepackage[$spacing$spacing]{setspace}

%% ----------------------------------------------------------------------------
%%
%% Floating-Environments
%%
%% ----------------------------------------------------------------------------

\deflength\textfloatsep{12.0pt plus 2.0pt minus 2.0pt}

%% BUG: der AtBeginEnvironment-Krempel unten sorgt dafür, dass "wegfloatende"
%% Umgebungen Leerzeilen hinterlassen. Keine Ahnung, warum. WTF...

%% ----------------------------------------------------------------------------
%%
%% Figure-Environments
%%
%% ----------------------------------------------------------------------------

% \AtBeginEnvironment{figure}{\microtypesetup{activate=false} \singlespacing}

%% ----------------------------------------------------------------------------
%%
%% Listings-Environments
%%
%% ----------------------------------------------------------------------------

% \AtBeginEnvironment{listings}{\microtypesetup{activate=false} \singlespacing}

%% ----------------------------------------------------------------------------
%%
%% Verbatim-Environments
%%
%% ----------------------------------------------------------------------------

% \AtBeginEnvironment{verbatim}{\microtypesetup{activate=false}}

%% ----------------------------------------------------------------------------
%%
%% Tabellen
%%
%% ----------------------------------------------------------------------------

%% Tabular figures für Tabellen ;-)
\AtBeginEnvironment{longtable}{\addfontfeatures{Numbers={Lowercase,Monospaced}} \singlespacing}
\AtBeginEnvironment{tabular}{\addfontfeatures{Numbers={Lowercase,Monospaced}}}

%% ----------------------------------------------------------------------------
%%
%% Mikrotypographische Feinheiten
%%
%% ----------------------------------------------------------------------------

%% Th - Ligaturen ausschalten
% \DisableLigatures[T]{encoding = *, family = rm* }
\usepackage[ngerman]{selnolig} % load selnolig w/o a language option
\nolig{Th}{T|h} % disable "Th" ligature globally

%% Aufrechte Klammern in kursivem Text
% \usepackage[biblatex=true]{embrac}
% \providecommand\textsi[1]{#1} % s. https://www.mail-archive.com/lyx-users@lists.lyx.org/msg110332.html
% % Einstellungen für Skolar
% \AddEmph{[}{]}[.04em,0.8em]

%% ----------------------------------------------------------------------------
%%
%% Seite
%%
%% ----------------------------------------------------------------------------

%% Seitenzahl
\addtokomafont{pagenumber}{%
    \addfontfeatures{Numbers={Tabular, Lining}, Letters={Uppercase}}\rmfamily\mdseries}

%% ----------------------------------------------------------------------------
%% Überschriften
%% ----------------------------------------------------------------------------

%%  Abschnittsnummerierung in den linken Rand ausrücken.
%%  Komascript definiert zur Ausgabe der Nummern pro
%%  Abschnittsart NAME den Befehl \NAMEformat, der hier
%%  umdefiniert wird. \theNAME gibt die Nummer selber aus,
%%  \autodot fügt je nach Konfiguration einen Punkt an.
%%  s. dazu auch scrguide Kap. 21
% \renewcommand*{\sectionformat}{%
%     \makebox[0pt][r]{\rmfamily\mdseries\upshape\thesection\autodot\enskip}}
% \renewcommand*{\subsectionformat}{%
%     \makebox[0pt][r]{\rmfamily\mdseries\upshape\thesubsection\autodot\enskip}}
% \renewcommand*{\subsubsectionformat}{%
%     \makebox[0pt][r]{\rmfamily\mdseries\itshape\thesubsubsection\autodot\enskip}}
% \renewcommand*{\paragraphformat}{%
%     \makebox[0pt][r]{\theparagraph\autodot\enskip}}
% \renewcommand*{\paragraphformat}{%
%     \makebox[0pt][r]{\thesubparagraph\autodot\enskip}}

% \renewcommand*{\chapterformat}{%
%     \makebox[1cm][l]{\rmfamily\bfseries\Large\thechapter\autodot\enskip}}
% \renewcommand*{\sectionformat}{%
%     \makebox[1cm][l]{\rmfamily\bfseries\large\thesection\autodot\enskip}}
% \renewcommand*{\subsectionformat}{%
%     \makebox[1cm][l]{\rmfamily\bfseries\thesubsection\autodot\enskip}}
% \renewcommand*{\subsubsectionformat}{%
%     \makebox[1cm][l]{\rmfamily\bfseries\thesubsubsection\autodot\enskip}}

\renewcommand*{\chapterformat}{%
    \makebox[0pt][r]{\rmfamily\bfseries\Large\thechapter\autodot\enskip}}
\renewcommand*{\sectionformat}{%
    \makebox[0pt][r]{\rmfamily\bfseries\large\thesection\autodot\enskip}}
\renewcommand*{\subsectionformat}{%
    \makebox[0pt][r]{\rmfamily\bfseries\thesubsection\autodot\enskip}}
\renewcommand*{\subsubsectionformat}{%
    \makebox[0pt][r]{\rmfamily\bfseries\thesubsubsection\autodot\enskip}}


%% Überschrifttext
\addtokomafont{chapter}{%
    \rmfamily\bfseries\Large}
\addtokomafont{section}{%
    \rmfamily\bfseries\large}
\addtokomafont{subsection}{%
    \rmfamily\bfseries}
\addtokomafont{subsubsection}{%
    \rmfamily\bfseries}
%\addtokomafont{disposition}{%
%    \color{darkgray}}

%% Abstände um Überschriften ändern
%% s. http://texwelt.de/wissen/fragen/10289/wie-andere-ich-die-abstande-uberunter-section-subsection-subsubsection
%\RedeclareSectionCommand[%
  %beforeskip=-1em,%
  %afterskip=1sp]{section}
%\RedeclareSectionCommand[%
  %beforeskip=-1em,%
  %afterskip=1sp]{subsection}
%\RedeclareSectionCommand[%
  %beforeskip=-1em,%
  %afterskip=1sp]{subsubsection}

% \RedeclareSectionCommands[%
%   beforeskip=-1\baselineskip,%
%   afterskip=1sp%
% ]{chapter,section,subsection,subsubsection}
\RedeclareSectionCommands[%
  beforeskip=-1\baselineskip,%
  afterskip=1sp%
]{chapter,section,subsection}
\RedeclareSectionCommand[%
  beforeskip=-1em,%
  afterskip=-0.5em%
]{subsubsection}

%% ----------------------------------------------------------------------------
%%
%% Inhaltsverzeichnis
%%
%% ----------------------------------------------------------------------------

\DeclareTOCStyleEntries[
    entrynumberformat   = {\normalfont\normalsize},
    % pagenumberformat    = {~—~\normalfont\normalsize},
    pagenumberformat    = {\normalfont\normalsize},
    raggedentrytext     = {true},
    % beforeskip          = {1sp},
    % numwidth            = {3em},
    % linefill            = {}
    linefill            = {\TOCLineLeaderFill[ ]}
]{tocline}{chapter, section, subsection, subsubsection, figure, table}

\DeclareTOCStyleEntry[
    entryformat         = {\bfseries\normalsize},
    pagenumberformat    = {\bfseries\normalsize},
    % beforeskip          = {1em plus 1pt},
    % linefill            = {}
    linefill            = {\TOCLineLeaderFill[.]}
]{tocline}{chapter}

\DeclareTOCStyleEntries[
    entryformat         = {\normalfont\normalsize}
]{tocline}{section, subsection, subsubsection}

\DeclareTOCStyleEntries[
    dynindent           = {false},
    dynnumwidth         = {false},
    indent              = {1sp},
    % entrynumberformat   = {\normalfont\normalsize\itshape},
    % entryformat         = {\normalfont\normalsize\itshape},
    linefill            = {\TOCLineLeaderFill[.]}
]{tocline}{figure, table}

%% ----------------------------------------------------------------------------
%%
%% Captions
%%
%% ----------------------------------------------------------------------------
% \addtokomafont{caption}{%
%     \rmfamily\mdseries\small}
% \addtokomafont{captionlabel}{%
%     \rmfamily\bfseries\small}

\captionsetup{%
    textfont={rm, small},
    labelfont={rm, bf, small},
}

%% ----------------------------------------------------------------------------
%%
%% Fußnoten
%%
%% ----------------------------------------------------------------------------

% %% Ziffern linksbündig, Einzug gleich Markenbreite
% \deffootnote[1.5em]{1.5em}{1em}{%
%     \makebox[1.5em][l]{\thefootnotemark\ }}

%% Ziffern linksbündig, kein Einzug
\deffootnote[1.5em]{1sp}{1em}{%
    \makebox[1.5em][l]{\thefootnotemark\ }}

%% Schriftart der Fußnotenmarkierung im Text
\setkomafont{footnotereference}{%
    \addfontfeatures{Numbers={Uppercase,Proportional}}}

%% Schriftart der Fußnotenmarke
\setkomafont{footnotelabel}{%
    \rmfamily\bfseries\addfontfeatures{Numbers={Uppercase,Proportional}}}

%% ----------------------------------------------------------------------------
%%
%% Definiert \textuppercase - Text in GROẞBUCHSTABEN mit Tracking
%%
%% ----------------------------------------------------------------------------
\newcommand{\textuppercase}[1]{%
{%
    \addfontfeatures{Numbers={Uppercase,Proportional}}%
    \microtypecontext{tracking=allcaps}%
    \lsstyle\MakeUppercase{#1}}}%
% \makeatletter
%     \renewcommand{\sectionlinesformat}[4]{%
%     \Ifstr{#1}{section}{%
%         \@hangfrom{\hskip #2#3}{\textuppercase{#4}}% War einen ganzen Tag Fehlersuche wert: so zerschießt es einem die Hyperrefs auf unnummerierte Sections...
%     }{%
%         \@hangfrom{\hskip #2#3}{#4}%
%     }%
% }
% \makeatother
% \makeatletter
% \renewcommand{\sectionlinesformat}[4]{%
%     \@hangfrom{\hskip #2#3}{#4}}
% \makeatother

%% ----------------------------------------------------------------------------
%%
%% Literaturverzeichnis
%%
%% ----------------------------------------------------------------------------

%% Familienname in Kapitälchen
\renewcommand{\mkbibnamefamily}[1]{\textsc{#1}}

%% Raum zwischen Einträgen, kein Einzug
\setlength\bibitemsep{0.4\baselineskip}
\setlength\bibhang{1sp}

%% DOI in Dokumentschrift
\DeclareFieldFormat{doi}{%
    \mkbibacro{DOI}\addcolon\space \ifhyperref {\href{https://doi.org/#1}{\normalfont #1}} {\normalfont #1}}

